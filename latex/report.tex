\documentclass[10pt,twocolumn,letterpaper]{article}

\usepackage[labelfont=bf]{caption}
\usepackage[table]{xcolor}
\usepackage{graphicx}
\usepackage{multirow}
\usepackage{amsfonts}
\usepackage{amssymb}
\usepackage[tbtags]{amsmath}
\usepackage[a4paper,margin=4cm]{geometry}
\usepackage{lastpage}
\usepackage{fancyhdr}
\usepackage[round]{natbib}
\usepackage{listings}
\usepackage{color}
\usepackage[title]{appendix}
\usepackage{bm}
\usepackage{textcomp}
\usepackage{mathtools}
\usepackage{tabularx}
\usepackage{booktabs}
\usepackage{ragged2e}
\usepackage{float}
\usepackage{enumitem}
\usepackage{tabularx}

\bibliographystyle{plainnat}
\def\sumin{\sum_{i=1}^{n}}
\def\bhline{\noalign{\hrule height 1pt}}

\DeclarePairedDelimiter\abs{\lvert}{\rvert}

%table
\definecolor{faintgrey}{RGB}{242, 242, 242}
\def\evenrow{\rowcolor{faintgrey}}
\captionsetup[table]{skip=0pt}
\newcolumntype{L}{>{\raggedright\arraybackslash}X}

% Include other packages here, before hyperref.

% If you comment hyperref and then uncomment it, you should delete
% egpaper.aux before re-running latex.  (Or just hit 'q' on the first latex
% run, let it finish, and you should be clear).
\usepackage[breaklinks=true,bookmarks=false]{hyperref}

\def\httilde{\mbox{\tt\raisebox{-.5ex}{\symbol{126}}}}

% Pages are numbered in submission mode, and unnumbered in camera-ready
%\ifcvprfinal\pagestyle{empty}\fi
%\setcounter{page}{4321}
\begin{document}

%%%%%%%%% TITLE
\title{\LaTeX\ Author Guidelines for CVPR Proceedings}

\author{Wasin Pipattungsakul \\
The University of Adelaide \\
{\tt\small wasin.pipattungsakul@adelaide.edu.au}
}

\maketitle
%\thispagestyle{empty}

%%%%%%%%% ABSTRACT
\begin{abstract}
  The ABSTRACT is to be in fully-justified italicized text, at the top
  of the left-hand column, below the author and affiliation
  information. Use the word ``Abstract'' as the title, in 12-point
  Times, boldface type, centered relative to the column, initially
  capitalized. The abstract is to be in 10-point, single-spaced type.
  Leave two blank lines after the Abstract, then begin the main text.
  Look at previous CVPR abstracts to get a feel for style and length.
\end{abstract}

%%%%%%%%% BODY TEXT
\section{Introduction}

This work shows the implementation of a perceptron algorithm to identify people with diabetes from a publicly
available dataset \citep{data}. The baseline model used to compare the performance of the perceptron algorithm
was a logistic regression model. For simplicity, only accuracy was used as a metric to compare the performance
of both models.

%-------------------------------------------------------------------------
\subsection{Diabetes Dataset}

According to \citet{data}, the dataset originates from the National Institute of Diabetes and Digestive and
Kidney Diseases. The dataset consists of 768 records of 8 features and 1 target label. The features are:

\begin{itemize}
  \item Pregnancies: Number of times pregnant
  \item Glucose: Plasma glucose concentration
  \item BloodPressure: Diastolic blood pressure (mm Hg)
  \item SkinThickness: Triceps skin fold thickness (mm)
  \item Insulin: 2-hour serum insulin (µU/ml)
  \item BMI: Body mass index (kg/m\textsuperscript{2})
  \item DiabetesPedigreeFunction: Diabetes pedigree function
  \item Age: Age (years)
\end{itemize}

The target label is 1 if the person has diabetes and 0 otherwise.

%-------------------------------------------------------------------------
\subsection{Perceptron}

A perceptron is an artificial neuron designed to mimic how a neuron might work \citep{rosenblatt-1958}. It can
be used to classify linearly separable data. The implementation of the perceptron algorithm is explained later
in Methods section.


%------------------------------------------------------------------------
\section{Methods}


%-------------------------------------------------------------------------
\subsection{Data}

Diabetes data was split into:

\begin{itemize}
  \item Training set: 537 records
  \item Validation set: 115 records
  \item Test set: 116 records
\end{itemize}

Then, all numerical features were normalized according to those in the training set. Also, before training the
model, the data was briefly explored to understand if there were any missing values. They would be imputed using
the means of the feature in the training set.

%-------------------------------------------------------------------------
\subsection{Perceptron Algorithm}

The perceptron algorithm implemented here consists of 2 parts, net input and activation function
\citep{perceptron}. The net input can be calculated as
\begin{align}
  \text{net}_i &= \sum_{j=1}^{m} w_j x_{ij} - \theta  \\
  &= \sum_{j=0}^{m} w_j x_{ij}, x_{i0} = 1, w_0 = -\theta \\
  &= \bm{w}^T\bm{x}_i \label{eq:net-input}
\end{align}
where \( x_{ij} \) is a \( j \)th input feature of an \( i \)th record, \( w_j \) is the \( j \)th weight of the
algorithm, \( m \) is the number of features, and \( \theta\in\mathbb{R} \). An acitvation function \( \phi \)
used is a threshold function
\begin{align}
  \phi(x) &= \left\{
    \begin{array}{ll}
      1, & x \ge 0 \\
      0, & x < 0
    \end{array}
  \right.\rlap{.} \label{eq:threshold-function}
\end{align}

The weights are randomly initialised (\( \bm{w}_0 \)) and updated with the following equations
\begin{align}
  \hat{y}_{t,i} &= \phi(\text{net}_{t,i}) \\
  e_{t,i} &= y_i - \hat{y}_{t,i} \\
  w_{t+1,j} &= w_{t,j} + \eta\sumin x_{ij}e_{t,i} \\
  \bm{w}_{t+1} &= \bm{w}_t + \eta\sumin\bm{x}_i e_{t,i}
\end{align}
where $\hat{y}_{t,i}$ is the prediction of the $i$th record at the $t$th iteration, $e_{t,i}$ is the error of
the $i$th record at the $t$th iteration, $y_i$ is the target label of the $i$th record, $\bm{w}_t$ is the
weights at the $t$th iteration, and $\eta$ is the learning rate for each iteration. This learning rate $\eta$
and the number of iterations are hyperparameters which can be tuned later.

%-------------------------------------------------------------------------
\subsection{Model Training}

For all perceptron models, the hyperparameters were tuned using the training and validation sets according to
table \ref{tab:hyperparameters}.
\begin{table}[!ht]
  \centering
  \caption{Hyperparameters used to train and tune the models}
  \label{tab:hyperparameters}
  \begin{tabularx}{\columnwidth}{p{60pt}L}
    \rowcolor{lightgray}
    \bf Parameter         & \bf Value(s) \\
    \bhline
    Number of iterations  & 10,000, stopping after no improvement in 20 iterations \\
    \evenrow
    Learning rate         & [0.1, 0.01, 0.001]
  \end{tabularx}
\end{table}
The best model was then chosen based on the accuracy from the validation set follow the formula
\begin{align}
  \text{accuracy} = 1 - \frac{\sumin\abs{y_i - \hat{y}_i}}{n}
\end{align}
where $y_i$ is the target label and $\hat{y}_i$ is the prediction of an $i$th record in the validation set.
Then, the accuracy of the best model could be compared with that of the logistic regression model.

If any imputation is required, the whole training process is executed again to compare the accuracy of the
models after the imputation.

%-------------------------------------------------------------------------
\section{Code}

The implementation can be found \href{https://github.com/wvjgsuhp/deep-learning-fundamentals-ass-1}{here}.

%-------------------------------------------------------------------------
\section{Results}

%-------------------------------------------------------------------------
\subsection{Exploratory Analysis}

Table \ref{tab:missing-values} shows that there are no missing values for both features and target in the
dataset.
However, there are zeros shown in table which is not realistic for the following features:
\begin{itemize}
  \item A glucose below 40 mg/dL is fatal \citep{glucose}.
  \item A blood pressure of 0 mm Hg is fatal.
  \item Zero skin thickness is not possible.
  \item Zero BMI indicates that the person weighs 0 kg which is not possible.
\end{itemize}
There are many causes for these zeros, for example, incorrect reading and incorrect data entry. These values
would be imputed using the means of the respective feature from the training set.
\noindent
\begin{table}[!ht]
  \caption{The number of missing and zero values in each feature}\label{tab:missing-values}
  \begin{center}
    \begin{tabularx}{\columnwidth}{p{81.5pt}LL}
      \rowcolor{lightgray}
      \textbf{Feature} & \textbf{\# of missings} & \textbf{\# of zeros} \\
      \bhline
      Pregnancies               & 0 & 111 \\
      \evenrow
      Glucose                   & 0 & 5 \\
      BloodPressure             & 0 & 35 \\
      \evenrow
      SkinThickness             & 0 & 227 \\
      Insulin                   & 0 & 374 \\
      \evenrow
      BMI                       & 0 & 11 \\
      DiabetesPedigree-Function & 0 & 0 \\
      \evenrow
      Age                       & 0 & 0 \\
      Outcome                   & 0 & 500
    \end{tabularx}
  \end{center}
\end{table}

%-------------------------------------------------------------------------
\subsection{Model Performance}

\noindent
\begin{table}[!ht]
  \centering
  \caption{The performance of the models}
  \label{tab:model-performance}
  \begin{tabularx}{\columnwidth}{p{50pt}LLL}
    \rowcolor{lightgray}
    \bf Model & \multicolumn{3}{c}{\bf Accuracy} \\
    \rowcolor{lightgray}
              & \bf Train & \bf Valida-tion & \bf Test \\
    \bhline
    \multicolumn{4}{c}{\bf Without imputation} \\
    \hline
    Logistic regression & 0.801 & 0.704 & 0.741 \\
    \evenrow
    Perceptron          & 0.771 & 0.704 & 0.716 \\
    \hline
    \multicolumn{4}{c}{\bf With imputation} \\
    \hline
    Logistic regression & 0.793 & 0.722 & 0.724 \\
    \evenrow
    Perceptron          & 0.782 & 0.757 & 0.724 \\
  \end{tabularx}
\end{table}
From table \ref{tab:model-performance}, a logistic regression model has the same validation accuracy (0.704) as
a perceptron model without imputation. Thus, the logistic regression model would be chosen due to being simpler
As an observation, its performance is also better on the test set (0.741 against 0.716).

For the accuracy of the models after imputation, a perceptron model has considerably better accuracy (0.757)
compared with that of a logistic regression model (0.722). Nevertheless, the observed accuracies (0.724) are the
same for both models.

%-------------------------------------------------------------------------
\section{Conclusion}

A perceptron algorithm can be used as a classification model and has comparable accuracy compared with a
logistic regression model. Additionally, the quality of the data highly affects the performance of the model. To
extend this work, we could
\begin{enumerate}
  \item measure the performance with different metrics,
  \item use several larger datasets,
  \item experiment with different activation functions, or
  \item apply transformations to the features.
\end{enumerate}

%------------------------------------------------------------------------
\bibliography{references.bib}

\end{document}

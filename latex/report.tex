\documentclass[10pt,twocolumn,letterpaper]{article}

\usepackage[labelfont=bf]{caption}
\usepackage[table]{xcolor}
\usepackage{graphicx}
\usepackage{multirow}
\usepackage{amsfonts}
\usepackage{amssymb}
\usepackage{amsmath}
\usepackage[a4paper,margin=4cm]{geometry}
\usepackage{lastpage}
\usepackage{fancyhdr}
\usepackage[round]{natbib}
\usepackage{listings}
\usepackage{color}
\usepackage[title]{appendix}
\usepackage{bm}
\usepackage{textcomp}
\usepackage{mathtools}
\usepackage{tabularx}
\usepackage{booktabs}
\usepackage{ragged2e}
\usepackage{float}
\usepackage{enumitem}

\bibliographystyle{plainnat}

% Include other packages here, before hyperref.

% If you comment hyperref and then uncomment it, you should delete
% egpaper.aux before re-running latex.  (Or just hit 'q' on the first latex
% run, let it finish, and you should be clear).
\usepackage[breaklinks=true,bookmarks=false]{hyperref}

\def\httilde{\mbox{\tt\raisebox{-.5ex}{\symbol{126}}}}

% Pages are numbered in submission mode, and unnumbered in camera-ready
%\ifcvprfinal\pagestyle{empty}\fi
%\setcounter{page}{4321}
\begin{document}

%%%%%%%%% TITLE
\title{\LaTeX\ Author Guidelines for CVPR Proceedings}

\author{Wasin Pipattungsakul \\
The University of Adelaide \\
{\tt\small wasin.pipattungsakul@adelaide.edu.au}
}

\maketitle
%\thispagestyle{empty}

%%%%%%%%% ABSTRACT
\begin{abstract}
   The ABSTRACT is to be in fully-justified italicized text, at the top
   of the left-hand column, below the author and affiliation
   information. Use the word ``Abstract'' as the title, in 12-point
   Times, boldface type, centered relative to the column, initially
   capitalized. The abstract is to be in 10-point, single-spaced type.
   Leave two blank lines after the Abstract, then begin the main text.
   Look at previous CVPR abstracts to get a feel for style and length.
\end{abstract}

%%%%%%%%% BODY TEXT
\section{Introduction}

Please \citep{gu-2021} follow the steps outlined below when submitting your manuscript to
the IEEE Computer Society Press.  This style guide now has several
important modifications (for example, you are no longer warned against the
use of sticky tape to attach your artwork to the paper), so all authors
should read this new version.

%-------------------------------------------------------------------------
\subsection{Perceptron}

\noindent

\begin{figure*}
\begin{center}
\fbox{\rule{0pt}{2in} \rule{.9\linewidth}{0pt}}
\end{center}
   \caption{Example of a short caption, which should be centered.}
\label{fig:short}
\end{figure*}

%------------------------------------------------------------------------
\section{Methods}


%-------------------------------------------------------------------------
\subsection{Data}


%-------------------------------------------------------------------------
\subsection{Model Training}


%-------------------------------------------------------------------------
\section{Code}


%-------------------------------------------------------------------------
\section{Results}


%-------------------------------------------------------------------------
\section{Conclusion}


%------------------------------------------------------------------------
\clearpage
\bibliography{references.bib}

\end{document}
